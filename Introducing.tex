



    
    \subsection{Các kiểu dữ liệu}\label{cuxe1c-kiux1ec3u-dux1eef-liux1ec7u}

    \subsubsection{Số nguyên}\label{sux1ed1-nguyuxean}

    
\begin{Verbatim}[commandchars=\\\{\}]
{\color{incolor}In [{\color{incolor}1}]:} \PY{n}{x} \PY{o}{=} \PY{l+m+mi}{20}
        \PY{n+nb}{print}\PY{p}{(}\PY{n+nb}{type}\PY{p}{(}\PY{n}{x}\PY{p}{)}\PY{p}{)}
\end{Verbatim}
    

    \begin{Verbatim}[commandchars=\\\{\}]
<class 'int'>

    \end{Verbatim}

    \subsubsection{Số thực}\label{sux1ed1-thux1ef1c}

    
\begin{Verbatim}[commandchars=\\\{\}]
{\color{incolor}In [{\color{incolor}2}]:} \PY{n}{x} \PY{o}{=} \PY{l+m+mf}{17.0}
        \PY{n+nb}{print}\PY{p}{(}\PY{n+nb}{type}\PY{p}{(}\PY{n}{x}\PY{p}{)}\PY{p}{)}
\end{Verbatim}
    

    \begin{Verbatim}[commandchars=\\\{\}]
<class 'float'>

    \end{Verbatim}

    \subsubsection{Số phức}\label{sux1ed1-phux1ee9c}

    
\begin{Verbatim}[commandchars=\\\{\}]
{\color{incolor}In [{\color{incolor}3}]:} \PY{n}{x} \PY{o}{=} \PY{l+m+mi}{20} \PY{o}{+} \PY{l+m+mi}{17}\PY{n}{j}
        \PY{n+nb}{print}\PY{p}{(}\PY{n+nb}{type}\PY{p}{(}\PY{n}{x}\PY{p}{)}\PY{p}{)}
        \PY{n+nb}{print}\PY{p}{(}\PY{n}{x}\PY{p}{)}
        \PY{n+nb}{print}\PY{p}{(}\PY{n}{x}\PY{o}{.}\PY{n}{real}\PY{p}{,} \PY{n}{x}\PY{o}{.}\PY{n}{imag}\PY{p}{)}
\end{Verbatim}
    

    \begin{Verbatim}[commandchars=\\\{\}]
<class 'complex'>
(20+17j)
20.0 17.0

    \end{Verbatim}

    \subsubsection{Logic}\label{logic}

    
\begin{Verbatim}[commandchars=\\\{\}]
{\color{incolor}In [{\color{incolor}4}]:} \PY{n}{x} \PY{o}{=} \PY{k+kc}{True}
        \PY{n+nb}{print}\PY{p}{(}\PY{n+nb}{type}\PY{p}{(}\PY{n}{x}\PY{p}{)}\PY{p}{)}
\end{Verbatim}
    

    \begin{Verbatim}[commandchars=\\\{\}]
<class 'bool'>

    \end{Verbatim}

    \subsubsection{Chuỗi}\label{chuux1ed7i}

    Chuỗi là cấu trúc lưu dữ liệu dạng văn bản (như tin nhắn chẳng hạn). Một
chuỗi có thể được gói giữa hai ký tự nháy kép (\texttt{"string"}) hoặc
nháy đơn (\texttt{\textquotesingle{}string\textquotesingle{}}). Khi sử
dụng nháy kép thì bên trong chuỗi có thể viết dấu nháy đơn dễ dàng hơn.

    
\begin{Verbatim}[commandchars=\\\{\}]
{\color{incolor}In [{\color{incolor}5}]:} \PY{n}{x} \PY{o}{=} \PY{l+s+s2}{\PYZdq{}}\PY{l+s+s2}{PiMA 2017 }\PY{l+s+s2}{\PYZsq{}}\PY{l+s+s2}{Math Modelling}\PY{l+s+s2}{\PYZsq{}}\PY{l+s+s2}{\PYZdq{}}
        \PY{n+nb}{print}\PY{p}{(}\PY{n+nb}{type}\PY{p}{(}\PY{n}{x}\PY{p}{)}\PY{p}{)}
\end{Verbatim}
    

    \begin{Verbatim}[commandchars=\\\{\}]
<class 'str'>

    \end{Verbatim}

    \subsubsection{Danh sách}\label{danh-suxe1ch}

    
\begin{Verbatim}[commandchars=\\\{\}]
{\color{incolor}In [{\color{incolor}6}]:} \PY{n}{x} \PY{o}{=} \PY{p}{[}\PY{l+m+mi}{1}\PY{p}{,} \PY{l+m+mi}{2}\PY{p}{,} \PY{l+m+mi}{3}\PY{p}{,} \PY{l+m+mi}{4}\PY{p}{]}
        \PY{n+nb}{print}\PY{p}{(}\PY{n+nb}{type}\PY{p}{(}\PY{n}{x}\PY{p}{)}\PY{p}{)}
\end{Verbatim}
    

    \begin{Verbatim}[commandchars=\\\{\}]
<class 'list'>

    \end{Verbatim}

    Một danh sách có thể bao gồm danh sách khác (nested list), các phần tử
của một danh sách có thể thuộc nhiều lớp dữ liệu khác nhau.

    
\begin{Verbatim}[commandchars=\\\{\}]
{\color{incolor}In [{\color{incolor}7}]:} \PY{n}{x} \PY{o}{=} \PY{p}{[}\PY{p}{[}\PY{l+m+mi}{1}\PY{p}{,} \PY{l+m+mi}{2}\PY{p}{]}\PY{p}{,} \PY{p}{[}\PY{l+m+mi}{3}\PY{p}{,} \PY{l+m+mi}{4}\PY{p}{]}\PY{p}{]} \PY{c+c1}{\PYZsh{} Ma trận 2x2}
        \PY{n}{x} \PY{o}{=} \PY{p}{[}\PY{p}{[}\PY{l+m+mf}{1.0}\PY{p}{,} \PY{l+m+mi}{2}\PY{p}{]}\PY{p}{,} \PY{l+m+mi}{3} \PY{o}{+} \PY{l+m+mi}{4}\PY{n}{j}\PY{p}{,} \PY{p}{[}\PY{l+m+mi}{5}\PY{p}{,} \PY{l+m+mi}{6}\PY{p}{]}\PY{p}{]}
        \PY{n+nb}{print}\PY{p}{(}\PY{n+nb}{type}\PY{p}{(}\PY{n}{x}\PY{p}{)}\PY{p}{)}
\end{Verbatim}
    

    \begin{Verbatim}[commandchars=\\\{\}]
<class 'list'>

    \end{Verbatim}

    \subsubsection{Từ điển}\label{tux1eeb-ux111iux1ec3n}

    
\begin{Verbatim}[commandchars=\\\{\}]
{\color{incolor}In [{\color{incolor}8}]:} \PY{n}{x} \PY{o}{=} \PY{p}{\PYZob{}}\PY{l+s+s2}{\PYZdq{}}\PY{l+s+s2}{P}\PY{l+s+s2}{\PYZdq{}}\PY{p}{:} \PY{l+s+s2}{\PYZdq{}}\PY{l+s+s2}{Project}\PY{l+s+s2}{\PYZdq{}}\PY{p}{,}
             \PY{l+s+s2}{\PYZdq{}}\PY{l+s+s2}{i}\PY{l+s+s2}{\PYZdq{}}\PY{p}{:} \PY{l+s+s2}{\PYZdq{}}\PY{l+s+s2}{in}\PY{l+s+s2}{\PYZdq{}}\PY{p}{,}
             \PY{l+s+s2}{\PYZdq{}}\PY{l+s+s2}{M}\PY{l+s+s2}{\PYZdq{}}\PY{p}{:} \PY{l+s+s2}{\PYZdq{}}\PY{l+s+s2}{Mathematics}\PY{l+s+s2}{\PYZdq{}}\PY{p}{,}
             \PY{l+s+s2}{\PYZdq{}}\PY{l+s+s2}{A}\PY{l+s+s2}{\PYZdq{}}\PY{p}{:} \PY{l+s+s2}{\PYZdq{}}\PY{l+s+s2}{Applications}\PY{l+s+s2}{\PYZdq{}}\PY{p}{\PYZcb{}}
        \PY{n+nb}{print}\PY{p}{(}\PY{n+nb}{type}\PY{p}{(}\PY{n}{x}\PY{p}{)}\PY{p}{)}
        \PY{n+nb}{print}\PY{p}{(}\PY{n}{x}\PY{p}{)}
\end{Verbatim}
    

    \begin{Verbatim}[commandchars=\\\{\}]
<class 'dict'>
\{'P': 'Project', 'i': 'in', 'M': 'Mathematics', 'A': 'Applications'\}

    \end{Verbatim}

    \subsubsection{Tuple}\label{tuple}

Trong Python, khi nhóm các giá trị lại với nhau thành cụm sẽ gọi là
\texttt{tuple}. Ví dụ như mỗi điểm trong không gian Oxyz được biểu diễn
bằng ba giá trị \((x; y; z)\) có thể được biểu diễn dạng tuple.

    
\begin{Verbatim}[commandchars=\\\{\}]
{\color{incolor}In [{\color{incolor}9}]:} \PY{n}{point} \PY{o}{=} \PY{p}{(}\PY{l+m+mi}{1}\PY{p}{,} \PY{l+m+mi}{2}\PY{p}{,} \PY{l+m+mi}{3}\PY{p}{)}
        \PY{n+nb}{print}\PY{p}{(}\PY{n+nb}{type}\PY{p}{(}\PY{n}{point}\PY{p}{)}\PY{p}{)}
        \PY{n}{x}\PY{p}{,} \PY{n}{y}\PY{p}{,} \PY{n}{z} \PY{o}{=} \PY{n}{point} \PY{c+c1}{\PYZsh{} Lấy ra giá trị từ tuple}
        \PY{n+nb}{print}\PY{p}{(}\PY{n}{x}\PY{p}{,} \PY{n}{y}\PY{p}{,} \PY{n}{z}\PY{p}{)}
        \PY{n}{y}\PY{p}{,} \PY{n}{x} \PY{o}{=} \PY{n}{x}\PY{p}{,} \PY{n}{y} \PY{c+c1}{\PYZsh{} Hoán đổi giá trị biến bằng tuple}
        \PY{n+nb}{print}\PY{p}{(}\PY{n}{x}\PY{p}{,} \PY{n}{y}\PY{p}{)}
\end{Verbatim}
    

    \begin{Verbatim}[commandchars=\\\{\}]
<class 'tuple'>
1 2 3
2 1

    \end{Verbatim}

    \subsection{Ép kiểu}\label{uxe9p-kiux1ec3u}

    Để ``ép'' một biến sang một kiểu dữ liệu khác có thể làm cú pháp như
sau:

    
\begin{Verbatim}[commandchars=\\\{\}]
{\color{incolor}In [{\color{incolor}10}]:} \PY{n}{x} \PY{o}{=} \PY{l+m+mi}{17} \PY{c+c1}{\PYZsh{} int}
         \PY{n}{y} \PY{o}{=} \PY{n+nb}{complex}\PY{p}{(}\PY{n}{x}\PY{p}{)}
         \PY{n+nb}{print}\PY{p}{(}\PY{n}{y}\PY{p}{,} \PY{n+nb}{type}\PY{p}{(}\PY{n}{y}\PY{p}{)}\PY{p}{)}
         \PY{n}{x} \PY{o}{=} \PY{l+m+mf}{5.3} \PY{c+c1}{\PYZsh{} float}
         \PY{n}{y} \PY{o}{=} \PY{n+nb}{int}\PY{p}{(}\PY{n}{x}\PY{p}{)}
         \PY{n+nb}{print}\PY{p}{(}\PY{n}{y}\PY{p}{,} \PY{n+nb}{type}\PY{p}{(}\PY{n}{y}\PY{p}{)}\PY{p}{)}
\end{Verbatim}
    

    \begin{Verbatim}[commandchars=\\\{\}]
(17+0j) <class 'complex'>
5 <class 'int'>

    \end{Verbatim}

    Tuy nhiên biến kiểu complex không thể chuyển sang float được mà phải
được lấy riêng phần thực và phần ảo.

    
\begin{Verbatim}[commandchars=\\\{\}]
{\color{incolor}In [{\color{incolor}11}]:} \PY{n}{x} \PY{o}{=} \PY{l+m+mi}{20} \PY{o}{+} \PY{l+m+mi}{17}\PY{n}{j}
         \PY{n}{a} \PY{o}{=} \PY{n+nb}{float}\PY{p}{(}\PY{n}{x}\PY{o}{.}\PY{n}{real}\PY{p}{)}
         \PY{n}{b} \PY{o}{=} \PY{n+nb}{int}\PY{p}{(}\PY{n}{x}\PY{o}{.}\PY{n}{imag}\PY{p}{)}
         \PY{n+nb}{print}\PY{p}{(}\PY{n}{a}\PY{p}{,} \PY{n}{b}\PY{p}{)}
\end{Verbatim}
    

    \begin{Verbatim}[commandchars=\\\{\}]
20.0 17

    \end{Verbatim}

    Biến kiểu string và hàm \texttt{range(begin,\ end,\ step)} có thể được
ép sang kiểu list.

    
\begin{Verbatim}[commandchars=\\\{\}]
{\color{incolor}In [{\color{incolor}12}]:} \PY{n}{x} \PY{o}{=} \PY{l+s+s2}{\PYZdq{}}\PY{l+s+s2}{PiMA}\PY{l+s+s2}{\PYZdq{}}
         \PY{n+nb}{print}\PY{p}{(}\PY{n+nb}{list}\PY{p}{(}\PY{n}{x}\PY{p}{)}\PY{p}{)}
         \PY{n+nb}{print}\PY{p}{(}\PY{n+nb}{range}\PY{p}{(}\PY{l+m+mi}{1}\PY{p}{,} \PY{l+m+mi}{7}\PY{p}{,} \PY{l+m+mi}{2}\PY{p}{)}\PY{p}{)}
         \PY{n+nb}{print}\PY{p}{(}\PY{n+nb}{list}\PY{p}{(}\PY{n+nb}{range}\PY{p}{(}\PY{l+m+mi}{1}\PY{p}{,} \PY{l+m+mi}{7}\PY{p}{,} \PY{l+m+mi}{2}\PY{p}{)}\PY{p}{)}\PY{p}{)}
\end{Verbatim}
    

    \begin{Verbatim}[commandchars=\\\{\}]
['P', 'i', 'M', 'A']
range(1, 7, 2)
[1, 3, 5]

    \end{Verbatim}

    \subsection{Phép toán}\label{phuxe9p-touxe1n}

    Python hỗ trợ các phép toán \texttt{+,\ -,\ *,\ /} (chia lấy kết quả
float), \texttt{//} (chia lấy phần nguyên, kết quả int), \texttt{**}
(lên lũy thừa).

Các phép biến đổi bit: \texttt{\&} (AND), \texttt{\textbar{}} (OR),
\texttt{\^{}} (XOR).

Đối với biến logic có \texttt{and}, \texttt{or}, \texttt{not}.

Các phép toán so sánh \texttt{\textless{}} (nhỏ hơn),
\texttt{\textgreater{}} (lớn hơn), \texttt{\textless{}=} (nhỏ hơn hoặc
bằng), \texttt{\textgreater{}=} (lớn hơn hoặc bằng), \texttt{==} (bằng)

    \subsubsection{Danh sách}\label{danh-suxe1ch}

    Các thao tác cơ bản cho danh sách bao gồm:

    \begin{itemize}
\tightlist
\item
  Thêm phần tử vào cuối list (\texttt{append})
\end{itemize}

    
\begin{Verbatim}[commandchars=\\\{\}]
{\color{incolor}In [{\color{incolor}13}]:} \PY{n}{x} \PY{o}{=} \PY{l+s+s2}{\PYZdq{}}\PY{l+s+s2}{Mthemmatic}\PY{l+s+s2}{\PYZdq{}}
         \PY{n}{x} \PY{o}{=} \PY{n+nb}{list}\PY{p}{(}\PY{n}{x}\PY{p}{)}
         \PY{n+nb}{print}\PY{p}{(}\PY{n}{x}\PY{p}{)}
         \PY{n}{x}\PY{o}{.}\PY{n}{append}\PY{p}{(}\PY{l+s+s1}{\PYZsq{}}\PY{l+s+s1}{s}\PY{l+s+s1}{\PYZsq{}}\PY{p}{)}
         \PY{n+nb}{print}\PY{p}{(}\PY{n}{x}\PY{p}{,} \PY{n+nb}{len}\PY{p}{(}\PY{n}{x}\PY{p}{)}\PY{p}{)}
\end{Verbatim}
    

    \begin{Verbatim}[commandchars=\\\{\}]
['M', 't', 'h', 'e', 'm', 'm', 'a', 't', 'i', 'c']
['M', 't', 'h', 'e', 'm', 'm', 'a', 't', 'i', 'c', 's'] 11

    \end{Verbatim}

    \begin{itemize}
\tightlist
\item
  Truy cập phần tử
\end{itemize}

Các phần tử trong list có thể được gọi dựa vào chỉ số (index) hay còn
gọi là thứ tự của nó trong danh sách. Trong một list, phần tử đầu tiên
có chỉ số là 0.

    
\begin{Verbatim}[commandchars=\\\{\}]
{\color{incolor}In [{\color{incolor}14}]:} \PY{n+nb}{print}\PY{p}{(}\PY{n}{x}\PY{p}{[}\PY{l+m+mi}{0}\PY{p}{]}\PY{p}{)}
\end{Verbatim}
    

    \begin{Verbatim}[commandchars=\\\{\}]
M

    \end{Verbatim}

    Một list còn có thể được truy cập thông qua một chỉ số âm. Đối với chỉ
số âm, phần tử cuối cùng trong danh sách được đánh số -1, phần tử liền
trước có chỉ số nhỏ hơn 1 đơn vị so với phần tử liền sau. Xem ví dụ:

    
\begin{Verbatim}[commandchars=\\\{\}]
{\color{incolor}In [{\color{incolor}15}]:} \PY{n+nb}{print}\PY{p}{(}\PY{n}{x}\PY{p}{)}
         \PY{n+nb}{print}\PY{p}{(}\PY{n}{x}\PY{p}{[}\PY{o}{\PYZhy{}}\PY{l+m+mi}{1}\PY{p}{]}\PY{p}{)}
         \PY{n+nb}{print}\PY{p}{(}\PY{n}{x}\PY{p}{[}\PY{o}{\PYZhy{}}\PY{l+m+mi}{2}\PY{p}{]}\PY{p}{)}
\end{Verbatim}
    

    \begin{Verbatim}[commandchars=\\\{\}]
['M', 't', 'h', 'e', 'm', 'm', 'a', 't', 'i', 'c', 's']
s
c

    \end{Verbatim}

    \begin{itemize}
\tightlist
\item
  Chèn phần tử vào một vị trí bất kỳ trong list (\texttt{insert})
\end{itemize}

    
\begin{Verbatim}[commandchars=\\\{\}]
{\color{incolor}In [{\color{incolor}16}]:} \PY{n}{x}\PY{o}{.}\PY{n}{insert}\PY{p}{(}\PY{l+m+mi}{1}\PY{p}{,} \PY{l+s+s1}{\PYZsq{}}\PY{l+s+s1}{a}\PY{l+s+s1}{\PYZsq{}}\PY{p}{)}
         \PY{n+nb}{print}\PY{p}{(}\PY{n}{x}\PY{p}{)}
\end{Verbatim}
    

    \begin{Verbatim}[commandchars=\\\{\}]
['M', 'a', 't', 'h', 'e', 'm', 'm', 'a', 't', 'i', 'c', 's']

    \end{Verbatim}

    \begin{itemize}
\tightlist
\item
  Xóa phần tử ở một vị trí bất kỳ (\texttt{del})
\end{itemize}

    
\begin{Verbatim}[commandchars=\\\{\}]
{\color{incolor}In [{\color{incolor}17}]:} \PY{k}{del}\PY{p}{(}\PY{n}{x}\PY{p}{[}\PY{l+m+mi}{5}\PY{p}{]}\PY{p}{)}
         \PY{n+nb}{print}\PY{p}{(}\PY{n}{x}\PY{p}{)}
\end{Verbatim}
    

    \begin{Verbatim}[commandchars=\\\{\}]
['M', 'a', 't', 'h', 'e', 'm', 'a', 't', 'i', 'c', 's']

    \end{Verbatim}

    \begin{itemize}
\tightlist
\item
  Xóa phần tử đầu tiên mang giá trị cho trước (\texttt{remove}).
\end{itemize}

    
\begin{Verbatim}[commandchars=\\\{\}]
{\color{incolor}In [{\color{incolor}18}]:} \PY{n}{x}\PY{o}{.}\PY{n}{remove}\PY{p}{(}\PY{l+s+s1}{\PYZsq{}}\PY{l+s+s1}{a}\PY{l+s+s1}{\PYZsq{}}\PY{p}{)}
         \PY{n+nb}{print}\PY{p}{(}\PY{n}{x}\PY{p}{)}
\end{Verbatim}
    

    \begin{Verbatim}[commandchars=\\\{\}]
['M', 't', 'h', 'e', 'm', 'a', 't', 'i', 'c', 's']

    \end{Verbatim}

    \begin{itemize}
\tightlist
\item
  Sắp xếp theo thứ tự tăng dần (\texttt{sort})
\end{itemize}

    
\begin{Verbatim}[commandchars=\\\{\}]
{\color{incolor}In [{\color{incolor}19}]:} \PY{n}{x}\PY{o}{.}\PY{n}{sort}\PY{p}{(}\PY{p}{)}
         \PY{n+nb}{print}\PY{p}{(}\PY{n}{x}\PY{p}{)}
\end{Verbatim}
    

    \begin{Verbatim}[commandchars=\\\{\}]
['M', 'a', 'c', 'e', 'h', 'i', 'm', 's', 't', 't']

    \end{Verbatim}

    \begin{itemize}
\tightlist
\item
  Cắt danh sách (slice) theo cú pháp
  \texttt{{[}begin\ :\ end\ :\ step{]}}.

  \begin{itemize}
  \tightlist
  \item
    \texttt{begin} mặc định là 0, \texttt{end} mặc định là độ dài của
    list, \texttt{step} mặc định là 1.
  \item
    Nếu cho \texttt{begin} là số âm thì \texttt{end} mặc định là 0.
  \item
    Nếu cho \texttt{end} là số âm thì \texttt{begin} mặc định là
    \texttt{-len(list)}.
  \item
    Nếu cho \texttt{step} là số âm thì mặc định
    \texttt{begin\ =\ len(list)-1} và \texttt{end\ =\ 0}.
  \item
    \ldots{}
  \end{itemize}
\end{itemize}

Các bạn có thể thoải mái khám phá thử nghiệm về trò chơi cắt mảng này.

    
\begin{Verbatim}[commandchars=\\\{\}]
{\color{incolor}In [{\color{incolor}20}]:} \PY{n+nb}{print}\PY{p}{(}\PY{n}{x}\PY{p}{[}\PY{l+m+mi}{2}\PY{p}{:}\PY{l+m+mi}{5}\PY{p}{]}\PY{p}{)}
         \PY{n+nb}{print}\PY{p}{(}\PY{n}{x}\PY{p}{[}\PY{p}{:}\PY{l+m+mi}{5}\PY{p}{]}\PY{p}{)}
         \PY{n+nb}{print}\PY{p}{(}\PY{n}{x}\PY{p}{[}\PY{l+m+mi}{5}\PY{p}{:}\PY{p}{]}\PY{p}{)}
         \PY{n+nb}{print}\PY{p}{(}\PY{n}{x}\PY{p}{[}\PY{p}{:}\PY{p}{:}\PY{o}{\PYZhy{}}\PY{l+m+mi}{1}\PY{p}{]}\PY{p}{)}
         \PY{n+nb}{print}\PY{p}{(}\PY{n}{x}\PY{p}{[}\PY{o}{\PYZhy{}}\PY{l+m+mi}{3}\PY{p}{:}\PY{p}{]}\PY{p}{)}
         \PY{n+nb}{print}\PY{p}{(}\PY{n}{x}\PY{p}{[}\PY{p}{:}\PY{o}{\PYZhy{}}\PY{l+m+mi}{4}\PY{p}{]}\PY{p}{)}
\end{Verbatim}
    

    \begin{Verbatim}[commandchars=\\\{\}]
['c', 'e', 'h']
['M', 'a', 'c', 'e', 'h']
['i', 'm', 's', 't', 't']
['t', 't', 's', 'm', 'i', 'h', 'e', 'c', 'a', 'M']
['s', 't', 't']
['M', 'a', 'c', 'e', 'h', 'i']

    \end{Verbatim}

    \subsubsection{Chuỗi}\label{chuux1ed7i}

    Chuỗi cũng có thao tác slice giống như list. Ngoài ra chuỗi có các hàm:

    \begin{itemize}
\tightlist
\item
  Thay thế một chuỗi con
\end{itemize}

    
\begin{Verbatim}[commandchars=\\\{\}]
{\color{incolor}In [{\color{incolor}21}]:} \PY{n}{x} \PY{o}{=} \PY{l+s+s2}{\PYZdq{}}\PY{l+s+s2}{PiMA 2xxx}\PY{l+s+s2}{\PYZdq{}}
         \PY{n}{x}\PY{o}{.}\PY{n}{replace}\PY{p}{(}\PY{l+s+s2}{\PYZdq{}}\PY{l+s+s2}{xxx}\PY{l+s+s2}{\PYZdq{}}\PY{p}{,} \PY{l+s+s2}{\PYZdq{}}\PY{l+s+s2}{017}\PY{l+s+s2}{\PYZdq{}}\PY{p}{)}
         \PY{n+nb}{print}\PY{p}{(}\PY{n}{x}\PY{p}{)}
         \PY{n}{x} \PY{o}{=} \PY{n}{x}\PY{o}{.}\PY{n}{replace}\PY{p}{(}\PY{l+s+s2}{\PYZdq{}}\PY{l+s+s2}{xxx}\PY{l+s+s2}{\PYZdq{}}\PY{p}{,} \PY{l+s+s2}{\PYZdq{}}\PY{l+s+s2}{017}\PY{l+s+s2}{\PYZdq{}}\PY{p}{)}
         \PY{n+nb}{print}\PY{p}{(}\PY{n}{x}\PY{p}{)}
\end{Verbatim}
    

    \begin{Verbatim}[commandchars=\\\{\}]
PiMA 2xxx
PiMA 2017

    \end{Verbatim}

    \begin{itemize}
\tightlist
\item
  Format chuỗi (đưa các dữ liệu khác vào giữa chuỗi theo kiểu C/C++)
\end{itemize}

    
\begin{Verbatim}[commandchars=\\\{\}]
{\color{incolor}In [{\color{incolor}22}]:} \PY{n}{x} \PY{o}{=} \PY{l+s+s2}{\PYZdq{}}\PY{l+s+s2}{PiMA }\PY{l+s+si}{\PYZpc{}d}\PY{l+s+s2}{\PYZdq{}} \PY{o}{\PYZpc{}} \PY{p}{(}\PY{l+m+mi}{2017}\PY{p}{)} \PY{c+c1}{\PYZsh{} \PYZpc{}d là số nguyên}
         \PY{n+nb}{print}\PY{p}{(}\PY{n}{x}\PY{p}{)}
         \PY{n+nb}{print}\PY{p}{(}\PY{l+s+s2}{\PYZdq{}}\PY{l+s+s2}{Pi = }\PY{l+s+si}{\PYZpc{}.2f}\PY{l+s+s2}{\PYZdq{}}\PY{p}{,} \PY{l+m+mf}{3.1415}\PY{p}{)} \PY{c+c1}{\PYZsh{} Làm tròn 2 chữ số thập phân}
         \PY{n+nb}{print}\PY{p}{(}\PY{l+s+s2}{\PYZdq{}}\PY{l+s+si}{\PYZob{}0\PYZcb{}}\PY{l+s+s2}{, }\PY{l+s+si}{\PYZob{}1\PYZcb{}}\PY{l+s+s2}{\PYZdq{}}\PY{o}{.}\PY{n}{format}\PY{p}{(}\PY{l+s+s2}{\PYZdq{}}\PY{l+s+s2}{Mathematics}\PY{l+s+s2}{\PYZdq{}}\PY{p}{,} \PY{l+m+mi}{2017}\PY{p}{)}\PY{p}{)}
         \PY{n+nb}{print}\PY{p}{(}\PY{l+s+s2}{\PYZdq{}}\PY{l+s+s2}{Math}\PY{l+s+s2}{\PYZdq{}} \PY{o}{+} \PY{l+s+s2}{\PYZdq{}}\PY{l+s+s2}{Modelling}\PY{l+s+s2}{\PYZdq{}}\PY{p}{)} \PY{c+c1}{\PYZsh{} Cộng chuỗi bằng phép tính}
         \PY{n+nb}{print}\PY{p}{(}\PY{l+s+s2}{\PYZdq{}}\PY{l+s+s2}{Math}\PY{l+s+s2}{\PYZdq{}}\PY{p}{,} \PY{l+s+s2}{\PYZdq{}}\PY{l+s+s2}{Modelling}\PY{l+s+s2}{\PYZdq{}}\PY{p}{,} \PY{l+m+mi}{2017}\PY{p}{)} \PY{c+c1}{\PYZsh{} Tự chèn khoảng cách}
\end{Verbatim}
    

    \begin{Verbatim}[commandchars=\\\{\}]
PiMA 2017
Pi = \%.2f 3.1415
Mathematics, 2017
MathModelling
Math Modelling 2017

    \end{Verbatim}

    \subsubsection{Từ điển}\label{tux1eeb-ux111iux1ec3n}

    Khi gọi một phần tử trong danh sách ta dùng chỉ số của phần tử (e.g.
\texttt{x{[}5{]}}). Tương tự trong từ điển, để tìm một giá trị (value)
ta dùng khóa (key). Quan hệ key-value giống với ánh xạ trong toán học.
Để thêm một ``định nghĩa'' mới (một bộ key-value mới) ta có thể thoải
mái ``gán'' (assign) mà không cần qua thao tác nào.

    
\begin{Verbatim}[commandchars=\\\{\}]
{\color{incolor}In [{\color{incolor}23}]:} \PY{n}{x} \PY{o}{=} \PY{p}{\PYZob{}}\PY{l+s+s2}{\PYZdq{}}\PY{l+s+s2}{P}\PY{l+s+s2}{\PYZdq{}}\PY{p}{:} \PY{l+s+s2}{\PYZdq{}}\PY{l+s+s2}{Project}\PY{l+s+s2}{\PYZdq{}}\PY{p}{,}
              \PY{l+s+s2}{\PYZdq{}}\PY{l+s+s2}{i}\PY{l+s+s2}{\PYZdq{}}\PY{p}{:} \PY{l+s+s2}{\PYZdq{}}\PY{l+s+s2}{in}\PY{l+s+s2}{\PYZdq{}}\PY{p}{,}
              \PY{l+s+s2}{\PYZdq{}}\PY{l+s+s2}{M}\PY{l+s+s2}{\PYZdq{}}\PY{p}{:} \PY{l+s+s2}{\PYZdq{}}\PY{l+s+s2}{Mathematics}\PY{l+s+s2}{\PYZdq{}}\PY{p}{,}
              \PY{l+s+s2}{\PYZdq{}}\PY{l+s+s2}{A}\PY{l+s+s2}{\PYZdq{}}\PY{p}{:} \PY{l+s+s2}{\PYZdq{}}\PY{l+s+s2}{Applications}\PY{l+s+s2}{\PYZdq{}}\PY{p}{\PYZcb{}}
         \PY{n+nb}{print}\PY{p}{(}\PY{l+s+s2}{\PYZdq{}}\PY{l+s+s2}{M stands for}\PY{l+s+s2}{\PYZdq{}}\PY{p}{,} \PY{n}{x}\PY{p}{[}\PY{l+s+s2}{\PYZdq{}}\PY{l+s+s2}{M}\PY{l+s+s2}{\PYZdq{}}\PY{p}{]}\PY{p}{)}
         \PY{n}{x}\PY{p}{[}\PY{l+s+s2}{\PYZdq{}}\PY{l+s+s2}{2017}\PY{l+s+s2}{\PYZdq{}}\PY{p}{]} \PY{o}{=} \PY{l+s+s2}{\PYZdq{}}\PY{l+s+s2}{the year PiMA was held}\PY{l+s+s2}{\PYZdq{}}
         \PY{n+nb}{print}\PY{p}{(}\PY{n}{x}\PY{p}{)}
\end{Verbatim}
    

    \begin{Verbatim}[commandchars=\\\{\}]
M stands for Mathematics
 \{'P': 'Project', 'i': 'in', 'M': 'Mathematics', 'A': 'Applications', '2017':
'the year PiMA was held'\}

    \end{Verbatim}

    \subsection{Vòng lặp và rẽ
nhánh}\label{vuxf2ng-lux1eb7p-vuxe0-rux1ebd-nhuxe1nh}

\subsubsection{\texorpdfstring{\texttt{for}}{for}}\label{for}

Về cơ bản, câu lệnh lặp \texttt{for} sẽ có cú pháp
\texttt{for\ \textless{}object\textgreater{}\ in\ \textless{}list-alike\textgreater{}}.
Trong đó \texttt{list-alike} có thể là hàm \texttt{items} của một dict,
có thể là hàm \texttt{range}, hoặc một list, hoặc một chuỗi, etc.

    
\begin{Verbatim}[commandchars=\\\{\}]
{\color{incolor}In [{\color{incolor}24}]:} \PY{k}{for} \PY{n}{element} \PY{o+ow}{in} \PY{p}{[}\PY{l+m+mi}{2}\PY{p}{,}\PY{l+m+mi}{0}\PY{p}{,}\PY{l+m+mi}{1}\PY{p}{,}\PY{l+m+mi}{7}\PY{p}{]}\PY{p}{:}
             \PY{k}{pass}
         \PY{k}{for} \PY{n}{key}\PY{p}{,} \PY{n}{value} \PY{o+ow}{in} \PY{n}{x}\PY{o}{.}\PY{n}{items}\PY{p}{(}\PY{p}{)}\PY{p}{:}
             \PY{n+nb}{print}\PY{p}{(}\PY{n}{key}\PY{p}{,} \PY{l+s+s2}{\PYZdq{}}\PY{l+s+s2}{\PYZhy{}\PYZhy{}\PYZgt{}}\PY{l+s+s2}{\PYZdq{}}\PY{p}{,} \PY{n}{value}\PY{p}{)}
         \PY{k}{for} \PY{n}{i} \PY{o+ow}{in} \PY{n+nb}{range}\PY{p}{(}\PY{l+m+mi}{1}\PY{p}{,} \PY{l+m+mi}{5}\PY{p}{,} \PY{l+m+mi}{2}\PY{p}{)}\PY{p}{:}
             \PY{k}{pass}
         \PY{k}{for} \PY{n}{character} \PY{o+ow}{in} \PY{l+s+s2}{\PYZdq{}}\PY{l+s+s2}{PiMA}\PY{l+s+s2}{\PYZdq{}}\PY{p}{:}
             \PY{k}{pass}
\end{Verbatim}
    

    \begin{Verbatim}[commandchars=\\\{\}]
P --> Project
i --> in
M --> Mathematics
A --> Applications
2017 --> the year PiMA was held

    \end{Verbatim}

    Ở trên, không có lỗi xảy ra nghĩa là các vòng lặp viết đúng cú pháp. Tuy
vậy, để ý rằng tôi có sử dụng lệnh \texttt{pass} để bỏ qua nội dung của
dòng \texttt{for} (tương tự với \texttt{while}, \texttt{if}, hay bất cứ
câu lệnh nào đòi hỏi một đoạn chương trình theo sau). \#\#\#
\texttt{while} Câu lệnh \texttt{while} có cú pháp
\texttt{while\ \textless{}conditions\textgreater{}:\ \textless{}statements\textgreater{}}.
Trong đó \texttt{conditions} có thể là một hoặc nhiều điều kiện ghép
nhau bằng \texttt{and}, \texttt{or}, \texttt{not}. Ví dụ sau liệt kê các
số chính phương nhỏ hơn 10.

    
\begin{Verbatim}[commandchars=\\\{\}]
{\color{incolor}In [{\color{incolor}25}]:} \PY{n}{i} \PY{o}{=} \PY{l+m+mi}{1}
         \PY{k}{while} \PY{p}{(}\PY{n}{i}\PY{o}{*}\PY{o}{*}\PY{l+m+mi}{2} \PY{o}{\PYZlt{}} \PY{l+m+mi}{10}\PY{p}{)}\PY{p}{:}
             \PY{n+nb}{print}\PY{p}{(}\PY{n}{i}\PY{o}{*}\PY{o}{*}\PY{l+m+mi}{2}\PY{p}{)}
             \PY{n}{i} \PY{o}{+}\PY{o}{=} \PY{l+m+mi}{1}
\end{Verbatim}
    

    \begin{Verbatim}[commandchars=\\\{\}]
1
4
9

    \end{Verbatim}

    \subsubsection{if}\label{if}

Câu lệnh \texttt{if} có cú pháp như sau:

\begin{verbatim}
if <conditions>:
    <statements>
else:
    <statements>
\end{verbatim}

Giả sử sau câu lệnh \texttt{else:} còn phải xét thêm điều kiện khác nữa
thì có thể thay bằng \texttt{elif\ \textless{}conditions:}.

    
\begin{Verbatim}[commandchars=\\\{\}]
{\color{incolor}In [{\color{incolor}26}]:} \PY{n}{x} \PY{o}{=} \PY{p}{[}\PY{l+m+mi}{1}\PY{p}{,}\PY{l+m+mi}{2}\PY{p}{]}
         \PY{n}{y} \PY{o}{=} \PY{p}{[}\PY{l+m+mi}{0}\PY{p}{,}\PY{l+m+mi}{3}\PY{p}{]}
         \PY{k}{if} \PY{n}{x} \PY{o}{\PYZgt{}} \PY{n}{y}\PY{p}{:}
             \PY{n+nb}{print}\PY{p}{(}\PY{l+s+s2}{\PYZdq{}}\PY{l+s+s2}{bigger}\PY{l+s+s2}{\PYZdq{}}\PY{p}{)}
         \PY{k}{elif} \PY{n}{x} \PY{o}{\PYZlt{}} \PY{n}{y}\PY{p}{:}
             \PY{n+nb}{print}\PY{p}{(}\PY{l+s+s2}{\PYZdq{}}\PY{l+s+s2}{smaller}\PY{l+s+s2}{\PYZdq{}}\PY{p}{)}
         \PY{k}{elif} \PY{n}{x} \PY{o}{==} \PY{n}{y}\PY{p}{:}
             \PY{n+nb}{print}\PY{p}{(}\PY{l+s+s2}{\PYZdq{}}\PY{l+s+s2}{equal}\PY{l+s+s2}{\PYZdq{}}\PY{p}{)}
         \PY{k}{else}\PY{p}{:}
             \PY{k}{pass} \PY{c+c1}{\PYZsh{} ???}
\end{Verbatim}
    

    \begin{Verbatim}[commandchars=\\\{\}]
bigger

    \end{Verbatim}

    \subsection{Hàm (function)}\label{huxe0m-function}

Để khai báo một hàm, dùng cú pháp:

    
\begin{Verbatim}[commandchars=\\\{\}]
{\color{incolor}In [{\color{incolor}27}]:} \PY{k}{def} \PY{n+nf}{f}\PY{p}{(}\PY{n}{param1}\PY{p}{,} \PY{n}{param2}\PY{o}{=}\PY{l+m+mi}{0}\PY{p}{)}\PY{p}{:}
             \PY{k}{return} \PY{n}{param1} \PY{o}{+} \PY{n}{param2} \PY{o}{*} \PY{l+m+mf}{0.5}
\end{Verbatim}
    

    Hàm \texttt{f} trên lấy hai tham số, trong đó tham số thứ hai
(\texttt{param2}) có giá trị mặc định là 0. Lưu ý rằng các tham số có
giá trị mặc định phải chuyển ra sau cùng trong danh sách tham số.

    
\begin{Verbatim}[commandchars=\\\{\}]
{\color{incolor}In [{\color{incolor}28}]:} \PY{k}{def} \PY{n+nf}{f2}\PY{p}{(}\PY{n}{param1}\PY{o}{=}\PY{l+m+mi}{0}\PY{p}{,} \PY{n}{param2}\PY{p}{)}\PY{p}{:}
             \PY{k}{return} \PY{n}{param1} \PY{o}{+} \PY{n}{param2}\PY{o}{*}\PY{l+m+mf}{0.5}
\end{Verbatim}
    


    \begin{Verbatim}[commandchars=\\\{\}]

          File "<ipython-input-28-9ce5ec1e6ed1>", line 1
        def f2(param1=0, param2):
              \string^
    SyntaxError: non-default argument follows default argument


    \end{Verbatim}


    Các cách sử dụng hàm sau đây là hợp lệ:

    
\begin{Verbatim}[commandchars=\\\{\}]
{\color{incolor}In [{\color{incolor}29}]:} \PY{n+nb}{print}\PY{p}{(}\PY{n}{f}\PY{p}{(}\PY{l+m+mi}{1}\PY{p}{)}\PY{p}{)}
         \PY{n+nb}{print}\PY{p}{(}\PY{n}{f}\PY{p}{(}\PY{l+m+mi}{1}\PY{p}{,}\PY{l+m+mi}{2}\PY{p}{)}\PY{p}{)}
         \PY{n+nb}{print}\PY{p}{(}\PY{n}{f}\PY{p}{(}\PY{n}{param2} \PY{o}{=} \PY{l+m+mi}{3}\PY{p}{,} \PY{n}{param1} \PY{o}{=} \PY{l+m+mi}{2}\PY{p}{)}\PY{p}{)}
\end{Verbatim}
    

    \begin{Verbatim}[commandchars=\\\{\}]
1.0
2.0
3.5

    \end{Verbatim}

    Như vậy khi gọi hàm, nếu không ghi rõ tên tham số truyền vào (cách 1, 2)
thì các giá trị gán lần lượt cho các tham số theo thứ tự khi khai báo
hàm. Nếu có ghi rõ (cách 3) thì không bắt buộc vị trí khai báo.

\subsubsection{\texorpdfstring{\texttt{lambda}}{lambda}}\label{lambda}

Thay vì khai báo hàm \texttt{f} bằng cách ghi \texttt{def\ f(x):} có thể
dùng phép gán cho những hàm đơn giản chỉ một dòng \texttt{return}.

    
\begin{Verbatim}[commandchars=\\\{\}]
{\color{incolor}In [{\color{incolor}30}]:} \PY{n}{f} \PY{o}{=} \PY{k}{lambda} \PY{n}{param1}\PY{p}{,} \PY{n}{param2}\PY{o}{=}\PY{l+m+mi}{0}\PY{p}{:} \PY{n}{param1} \PY{o}{+} \PY{n}{param2}\PY{o}{*}\PY{l+m+mf}{0.5}
         \PY{n+nb}{print}\PY{p}{(}\PY{n}{f}\PY{p}{(}\PY{l+m+mi}{1}\PY{p}{)}\PY{p}{)}
         \PY{n+nb}{print}\PY{p}{(}\PY{n}{f}\PY{p}{(}\PY{l+m+mi}{1}\PY{p}{,}\PY{l+m+mi}{2}\PY{p}{)}\PY{p}{)}
         \PY{n+nb}{print}\PY{p}{(}\PY{n}{f}\PY{p}{(}\PY{n}{param2}\PY{o}{=}\PY{l+m+mi}{3}\PY{p}{,} \PY{n}{param1}\PY{o}{=}\PY{l+m+mi}{2}\PY{p}{)}\PY{p}{)}
\end{Verbatim}
    

    \begin{Verbatim}[commandchars=\\\{\}]
1.0
2.0
3.5

    \end{Verbatim}

    \subsection{\texorpdfstring{Cách tạo nhanh
\texttt{list}}{Cách tạo nhanh list}}\label{cuxe1ch-tux1ea1o-nhanh-list}

    Một \texttt{list} có thể được tạo nhanh thông qua dòng \texttt{for}.
Chẳng hạn tôi muốn tạo một danh sách các số tự nhiên nhỏ hơn 10.

    
\begin{Verbatim}[commandchars=\\\{\}]
{\color{incolor}In [{\color{incolor}31}]:} \PY{n}{a} \PY{o}{=} \PY{p}{[}\PY{n}{x} \PY{k}{for} \PY{n}{x} \PY{o+ow}{in} \PY{n+nb}{range}\PY{p}{(}\PY{l+m+mi}{0}\PY{p}{,}\PY{l+m+mi}{10}\PY{p}{)}\PY{p}{]}
         \PY{n+nb}{print}\PY{p}{(}\PY{n}{a}\PY{p}{)}
\end{Verbatim}
    

    \begin{Verbatim}[commandchars=\\\{\}]
[0, 1, 2, 3, 4, 5, 6, 7, 8, 9]

    \end{Verbatim}

    Hoặc có thể được tạo nhanh qua hàm \texttt{lambda}. Chẳng hạn tôi muốn
tạo danh sách gồm bình phương của các số tự nhiên trong khoảng
\texttt{{[}4;\ 8{]}}:

    
\begin{Verbatim}[commandchars=\\\{\}]
{\color{incolor}In [{\color{incolor}32}]:} \PY{n}{a} \PY{o}{=} \PY{n+nb}{map}\PY{p}{(}\PY{k}{lambda} \PY{n}{x}\PY{p}{:} \PY{n}{x}\PY{o}{*}\PY{o}{*}\PY{l+m+mi}{2}\PY{p}{,} \PY{n+nb}{range}\PY{p}{(}\PY{l+m+mi}{4}\PY{p}{,} \PY{l+m+mi}{9}\PY{p}{)}\PY{p}{)} \PY{c+c1}{\PYZsh{} trả về con trỏ}
         \PY{n+nb}{print}\PY{p}{(}\PY{n}{a}\PY{p}{)} \PY{c+c1}{\PYZsh{} In ra địa chỉ}
         \PY{n+nb}{print}\PY{p}{(}\PY{n+nb}{list}\PY{p}{(}\PY{n}{a}\PY{p}{)}\PY{p}{)} \PY{c+c1}{\PYZsh{} Ép kiểu danh sách}
\end{Verbatim}
    

    \begin{Verbatim}[commandchars=\\\{\}]
<map object at 0x107eca550>
[16, 25, 36, 49, 64]

    \end{Verbatim}

    \subsection{\texorpdfstring{\texttt{*} Lớp
(\texttt{class})}{* Lớp (class)}}\label{lux1edbp-class}

    Class là chức năng quan trọng khiến Python trở thành ngôn ngữ lập trình
hướng đối tượng. Mỗi biến trong Python dù lớn hay nhỏ (\texttt{int}) đều
là một đối tượng (object). Trong một đối tượng chứa các thuộc tính
(attribute) đóng vai trò như biến để cung cấp thêm thông tin về đối
tượng đó, và các phương pháp (method) đóng vai trò như hàm để thực thi
một loạt các lệnh xử lý.

Ví dụ khi mô hình hóa xe cứu thương vào lập trình, có thể biến nó thành
một object với kiểu dữ liệu tên \texttt{Ambulance} được định nghĩa qua
lớp cùng tên. Xe cứu thương có thể bao gồm các thông số như biển số xe
(\texttt{string}) và vị trí hiện tại của xe (dựa theo mã tỉnh,
\texttt{int}). Thao tác có thể đặt bên trong đối tượng này là ``di
chuyển đến'' để hỗ trợ một xe cứu thương khác.

    
\begin{Verbatim}[commandchars=\\\{\}]
{\color{incolor}In [{\color{incolor}33}]:} \PY{k}{class} \PY{n+nc}{Ambulance}\PY{p}{:}
             \PY{k}{def} \PY{n+nf}{\PYZus{}\PYZus{}init\PYZus{}\PYZus{}}\PY{p}{(}\PY{n+nb+bp}{self}\PY{p}{,} \PY{n}{plate}\PY{p}{,} \PY{n}{city}\PY{p}{)}\PY{p}{:}
                 \PY{n+nb+bp}{self}\PY{o}{.}\PY{n}{plate} \PY{o}{=} \PY{n}{plate}
                 \PY{n+nb+bp}{self}\PY{o}{.}\PY{n}{city} \PY{o}{=} \PY{n}{city}
             \PY{k}{def} \PY{n+nf}{assist}\PY{p}{(}\PY{n+nb+bp}{self}\PY{p}{,} \PY{n}{another\PYZus{}ambulance}\PY{p}{)}\PY{p}{:}
                 \PY{n+nb+bp}{self}\PY{o}{.}\PY{n}{city} \PY{o}{=} \PY{n}{another\PYZus{}ambulance}\PY{o}{.}\PY{n}{city}
             \PY{k}{def} \PY{n+nf}{\PYZus{}\PYZus{}str\PYZus{}\PYZus{}}\PY{p}{(}\PY{n+nb+bp}{self}\PY{p}{)}\PY{p}{:}
                 \PY{k}{return} \PY{l+s+s2}{\PYZdq{}}\PY{l+s+s2}{Ambulance }\PY{l+s+si}{\PYZob{}0\PYZcb{}}\PY{l+s+s2}{ is in city }\PY{l+s+si}{\PYZob{}1\PYZcb{}}\PY{l+s+s2}{\PYZdq{}}\PYZbs{}
                         \PY{o}{.}\PY{n}{format}\PY{p}{(}\PY{n+nb+bp}{self}\PY{o}{.}\PY{n}{plate}\PY{p}{,} \PY{n+nb+bp}{self}\PY{o}{.}\PY{n}{city}\PY{p}{)}
\end{Verbatim}
    

    Lưu ý rằng trong mỗi \texttt{class} thường có hàm \texttt{\_\_init\_\_}
để khởi tạo một đối tượng. Trong ví dụ trên, hàm này có nhận tham số
\texttt{plate} và gán vào \texttt{self.plate}. Có thể hình dung rằng
\texttt{self.plate} là một biến \texttt{plate} là một thuộc tính (hay
biến) thuộc lớp Ambulance, còn tham số \texttt{plate} được truyền vào
hàm \texttt{\_\_init\_\_} là một biến độc lập chỉ có giá trị trong hàm
này chứ không thuộc lớp \texttt{Ambulance}. Hàm \texttt{\_\_str\_\_}
dùng để hướng dẫn cho câu lệnh \texttt{print} nên in ra thông tin gì khi
người ta cố gắng in đối tượng này ra màn hình.

Mọi hàm trong class đều phải có tham số \texttt{self}. Khi gọi hàm thuộc
class không truyền giá trị vào tham số này.

    
\begin{Verbatim}[commandchars=\\\{\}]
{\color{incolor}In [{\color{incolor}34}]:} \PY{n}{amb1} \PY{o}{=} \PY{n}{Ambulance}\PY{p}{(}\PY{l+s+s2}{\PYZdq{}}\PY{l+s+s2}{65\PYZhy{}B1\PYZhy{}2017}\PY{l+s+s2}{\PYZdq{}}\PY{p}{,} \PY{l+m+mi}{32}\PY{p}{)}
         \PY{n}{amb2} \PY{o}{=} \PY{n}{Ambulance}\PY{p}{(}\PY{l+s+s2}{\PYZdq{}}\PY{l+s+s2}{50\PYZhy{}B2\PYZhy{}11124}\PY{l+s+s2}{\PYZdq{}}\PY{p}{,} \PY{l+m+mi}{14}\PY{p}{)}
         \PY{n+nb}{print}\PY{p}{(}\PY{n}{amb1}\PY{p}{)}
         \PY{n+nb}{print}\PY{p}{(}\PY{n}{amb2}\PY{p}{)}
         \PY{n}{amb2}\PY{o}{.}\PY{n}{assist}\PY{p}{(}\PY{n}{amb1}\PY{p}{)}
         \PY{n+nb}{print}\PY{p}{(}\PY{n}{amb2}\PY{p}{)}
\end{Verbatim}
    

    \begin{Verbatim}[commandchars=\\\{\}]
Ambulance 65-B1-2017 is in city 32
Ambulance 50-B2-11124 is in city 14
Ambulance 50-B2-11124 is in city 32

    \end{Verbatim}

